% Created 2024-07-05 Fri 22:22
% Intended LaTeX compiler: pdflatex
\documentclass[11pt]{article}
\usepackage[utf8]{inputenc}
\usepackage[T1]{fontenc}
\usepackage{graphicx}
\usepackage{longtable}
\usepackage{wrapfig}
\usepackage{rotating}
\usepackage[normalem]{ulem}
\usepackage{amsmath}
\usepackage{amssymb}
\usepackage{capt-of}
\usepackage{hyperref}
\usepackage[a4paper,left=1cm,right=1cm,top=1cm,bottom=1cm]{geometry}
\usepackage[american, english]{babel}
\usepackage{enumitem}
\usepackage{float}
\usepackage[sc]{mathpazo}
\linespread{1.05}
\renewcommand{\labelitemi}{$\rhd$}
\setlength\parindent{0pt}
\setlist[itemize]{leftmargin=*}
\setlist{nosep}
\newcommand{\pv}{\mathrm{PV}}
\newcommand{\fv}{\mathrm{FV}}
\date{}
\title{Week 1 Quiz}
\hypersetup{
 pdfauthor={Marcio Woitek},
 pdftitle={Week 1 Quiz},
 pdfkeywords={},
 pdfsubject={},
 pdfcreator={Emacs 29.4 (Org mode 9.8)}, 
 pdflang={English}}
\begin{document}

\thispagestyle{empty}
\pagestyle{empty}
\section*{Problem 1}
\label{sec:orgcb0cc4d}

\textbf{Answer:} Opportunity Costs, Inflation, Risk
\section*{Problem 2}
\label{sec:org90feb7b}

\textbf{Answer:} \$1,500 at the end of 4 years\\

To answer this question, we need to know the present value of \$1,500. This value
can be computed with the aid of the formula
\begin{equation}
\pv=\frac{\fv}{(1+r)^t},
\end{equation}
where \(\fv\) denotes the future value, \(r\) denotes the interest rate, and
\(t\) is the number of time periods. In this case, we have
\begin{align}
  \begin{split}
    \fv&=1500,\\
    r&=0.1,\\
    t&=4.
  \end{split}
\end{align}
By substituting these values into the equation for \(\pv\), we get

\begin{equation}
\pv=\frac{1500}{(1+0.1)^4}\approx 1024.52.
\end{equation}
This result clearly shows that it's better to receive the prize at the end of 4
years.
\section*{Problem 3}
\label{sec:org673d4a4}

\textbf{Answer:} \$18,476.91\\

In this case, we have the following:
\begin{itemize}
\item the future value is \$20,000;
\item the interest rate is \(r=0.02\);
\item the number of time periods is \(t=4\).
\end{itemize}
Moreover, the amount needed today is the present value corresponding to \$20,000.
\(\pv\) can be calculated as follows:

\begin{align}
  \begin{split}
    \pv&=\frac{\fv}{(1+r)^t}\\
    &=\frac{20000}{(1+0.02)^4}\\
    &\approx 18476.91.
  \end{split}
\end{align}
\section*{Problem 4}
\label{sec:orgf695a08}

\textbf{Answer:} \(\pv\) goes up\\

We assume the future value and the interest rate are the same in both cases.
Then in the first case the present value is given by
\begin{equation}
\pv_1=\frac{\fv}{(1+r)^5}.
\end{equation}
If the number of periods changes from 5 to 4, the new present value is
\begin{equation}
\pv_2=\frac{\fv}{(1+r)^4}.
\end{equation}
To answer this question, we compute the ratio \(\frac{\pv_2}{\pv_1}\):
\begin{align}
  \begin{split}
    \frac{\pv_2}{\pv_1}&=\frac{\fv}{(1+r)^4}\frac{(1+r)^5}{\fv}\\
    &=\frac{(1+r)(1+r)^4}{(1+r)^4}\\
    &=1+r.
  \end{split}
\end{align}
The interest rate is a positive number. Then the above ratio is greater than 1.
Therefore, the new present value \(\pv_2\) is larger than the old one,
\(\pv_1\).
\section*{Problem 5}
\label{sec:org5525fa5}

\textbf{Answer:} \$1,628.90\\

In this case, we have the following:
\begin{itemize}
\item the present value is \$1,000;
\item the interest rate is \(r=0.05\);
\item the number of time periods is \(t=10\).
\end{itemize}
Moreover, the account balance represents the future value corresponding to
\$1,000. The future value can be determined with the aid of the formula
\begin{equation}
\fv=\pv(1+r)^t.
\end{equation}
By substituting the values of \(\pv\), \(r\) and \(t\) into the above
equation, we get

\begin{equation}
\fv=1000(1+0.05)^{10}\approx 1628.90.
\end{equation}
\section*{Problem 6}
\label{sec:org281acae}

\textbf{Answer:} \$1,638.62\\

In this case, we have the following:
\begin{itemize}
\item the present value is still \$1,000;
\item the interest rate is half of what we had in the previous problem, i.e.,
\(r=\frac{0.05}{2}=0.025\);
\item the number of time periods is double what we had in the previous problem,
i.e., \(t=2\cdot 10=20\).
\end{itemize}
Once again, we compute the account balance with the aid of the formula for the
future value:
\begin{equation}
\fv=\pv(1+r)^t.
\end{equation}
By substituting the values of \(\pv\), \(r\) and \(t\) into the above
equation, we get

\begin{equation}
\fv=1000(1+0.025)^{20}\approx 1638.62.
\end{equation}
\section*{Problem 7}
\label{sec:orge0c7860}

\textbf{Answer:} \$6,549.56\\

The value we want has two components. First, there's the present value
corresponding to the face value of \$6,000. We denote this amount by \(\pv_1\).
Then there's the present value corresponding to all the annual payments of \$300.
This amount is denoted by \(\pv_2\). Hence:
\begin{equation}
\pv=\pv_1+\pv_2.
\end{equation}
We perform the simplest calculation first. Since the discount rate is \(r=0.03\)
and the number of periods is \(t=5\), the PV related to the face value can be
computed as follows:

\begin{equation}
\pv_1=\frac{6000}{(1+0.03)^5}\approx 5175.65.
\end{equation}
Next, consider the value of \(\pv_2\). There will be 5 annual payments of
\$300. Then we can express \(\pv_2\) as

\begin{equation}
\pv_2=\sum_{t=1}^5\frac{300}{(1+0.03)^t}=300\sum_{t=1}^5\frac{1}{1.03^t}\approx 1373.91.
\end{equation}
To get the desired result, all we need to do now is to sum \(\pv_1\) and
\(\pv_2\). The present value of the bond is

\begin{equation}
\pv\approx 5175.65+1373.91\approx 6549.56.
\end{equation}
\section*{Problem 8}
\label{sec:orge97df85}

\textbf{Answer:} \$21.98\\

In this problem, we have the following:
\begin{itemize}
\item the earnings per share of company A is \(E_A=1.57\);
\item the P/E ratio for company B is \(\mathrm{PE}_B=14\).
\end{itemize}
The value of a share of stock in company A is denoted by \(P_A\). This value
can be computed as follows:

\begin{equation}
P_A=\mathrm{PE}_B\cdot E_A\approx 21.98.
\end{equation}
\section*{Problem 9}
\label{sec:org663b005}

\textbf{Answer:} \$913.22\\

The approach to solving this problem is the same we considered in Problem 7. So
we won't explain this approach again. We'll simply write down the final result.
To obtain this result, a Python function was implemented. The corresponding code
is presented below.
\begin{verbatim}
def bond_value(face_value: float, coupon: float, discount_rate: float, maturity: int) -> float:
    pv_1 = face_value / (1 + discount_rate) ** maturity
    pv_2 = coupon * sum((1 + discount_rate) ** (-t) for t in range(1, maturity + 1))
    pv = pv_1 + pv_2
    return pv
\end{verbatim}
To get the answer to this question, it's just a matter of calling the above
function with the right parameters:
\begin{verbatim}
face_value = 1000
coupon = 50
discount_rate = 0.1
maturity = 2
ans_9 = bond_value(face_value, coupon, discount_rate, maturity)
print(round(ans_9, 2))
\end{verbatim}
\phantomsection
\label{problem_9}
\begin{verbatim}
913.22
\end{verbatim}
\section*{Problem 10}
\label{sec:orge6b70f3}

\textbf{Answer:} \$1,038.27\\

To answer this question, we change the value of the discount rate, and then call
our function:
\begin{verbatim}
discount_rate = 0.03
ans_10 = bond_value(face_value, coupon, discount_rate, maturity)
print(round(ans_10, 2))
\end{verbatim}
\phantomsection
\label{problem_10}
\begin{verbatim}
1038.27
\end{verbatim}
\end{document}
