% Created 2024-07-09 Tue 08:34
% Intended LaTeX compiler: pdflatex
\documentclass[11pt]{article}
\usepackage[utf8]{inputenc}
\usepackage[T1]{fontenc}
\usepackage{graphicx}
\usepackage{longtable}
\usepackage{wrapfig}
\usepackage{rotating}
\usepackage[normalem]{ulem}
\usepackage{amsmath}
\usepackage{amssymb}
\usepackage{capt-of}
\usepackage{hyperref}
\usepackage[a4paper,left=1cm,right=1cm,top=1cm,bottom=1cm]{geometry}
\usepackage[american, english]{babel}
\usepackage{enumitem}
\usepackage{float}
\usepackage[sc]{mathpazo}
\linespread{1.05}
\renewcommand{\labelitemi}{$\rhd$}
\setlength\parindent{0pt}
\setlist[itemize]{leftmargin=*}
\setlist{nosep}
\date{}
\title{Week 4 Quiz}
\hypersetup{
 pdfauthor={Marcio Woitek},
 pdftitle={Week 4 Quiz},
 pdfkeywords={},
 pdfsubject={},
 pdfcreator={Emacs 29.4 (Org mode 9.8)}, 
 pdflang={English}}
\begin{document}

\thispagestyle{empty}
\pagestyle{empty}
\section*{Problem 1}
\label{sec:org79c82ef}

\textbf{Answer:} Stock A\\

\(\beta\) is a measure of price fluctuation for a given stock. Then a small
\(\beta\) indicates less fluctuations. This is precisely what a risk-averse
investor wants. Therefore, I would recommend \textbf{stock A}.
\section*{Problem 2}
\label{sec:org82376db}

\textbf{Answer:} 1.0\\

By definition, \(\beta\) is
\begin{equation}
\beta=\frac{\mathrm{Cov}(R_i,R_m)}{\mathrm{Var}(R_m)},
\end{equation}
where \(R_i\) denotes the rate of return for the stock of interest, and \(R_m\)
denotes the rate of return for some stock market index.\\
This question is about what happens when \(R_i=R_m\). By using the fact that
\(\mathrm{Cov}(X,X)=\mathrm{Var}(X)\), we can write \(\beta\) as follows:
\begin{align}
  \begin{split}
    \beta&=\frac{\mathrm{Cov}(R_i,R_m)}{\mathrm{Var}(R_m)}\\
    &=\frac{\mathrm{Cov}(R_m,R_m)}{\mathrm{Var}(R_m)}\\
    &=\frac{\mathrm{Var}(R_m)}{\mathrm{Var}(R_m)}\\
    &=1.
  \end{split}
\end{align}
Therefore, the value of \(\beta\) for the market is 1.
\section*{Problem 3}
\label{sec:org852ff12}

\textbf{Answer:} Firm A
\section*{Problem 4}
\label{sec:org4af14bb}

\textbf{Answer:} 21.25\%\\

We begin by summarizing the information we were given:
\begin{itemize}
\item The company's beta is \(\beta=1.5\);
\item the risk-free rate is 7\%, i.e., \(r_{\mathrm{RF}}=0.07\);
\item the equity premium is 9.5\%, i.e., \(\mathrm{EP}=0.095\).
\end{itemize}
As explained in the lectures, the cost of equity \(R_e\) can be computed as
follows:
\begin{equation}
R_e=r_{\mathrm{RF}}+\beta\cdot\mathrm{EP}.
\end{equation}
By substituting the known values into the RHS of the above equation, we get
\begin{align}
  \begin{split}
    R_e&=r_{\mathrm{RF}}+\beta\cdot\mathrm{EP}\\
    &=0.07+1.5\cdot 0.095\\
    &=0.2125.
  \end{split}
\end{align}
Therefore, the expected return is 21.25\%.
\section*{Problem 5}
\label{sec:orgbc74e54}

\textbf{Answer:} 0.89\\

To answer this question, we're going to use the equation for the cost of equity.
By solving this equation for \(\beta\), we get
\begin{align}
  \begin{split}
    R_e&=r_{\mathrm{RF}}+\beta\cdot\mathrm{EP}\\
    R_e-r_{\mathrm{RF}}&=\beta\cdot\mathrm{EP}\\
    \beta&=\frac{R_e-r_{\mathrm{RF}}}{\mathrm{EP}}
  \end{split}
\end{align}
We were given all the values on the RHS of the last equation:
\begin{itemize}
\item the expected return is 10.2\%, i.e., \(R_e=0.102\);
\item the risk-free rate is 4\%, i.e., \(r_{\mathrm{RF}}=0.04\);
\item the market risk premium is 7\%, i.e., \(\mathrm{EP}=0.07\).
\end{itemize}
By substituting these values into our formula for \(\beta\), we obtain
\begin{align}
  \begin{split}
    \beta&=\frac{R_e-r_{\mathrm{RF}}}{\mathrm{EP}}\\
    &=\frac{0.102-0.04}{0.07}\\
    &=0.89.
  \end{split}
\end{align}
Therefore, the value of beta for this stock is \(\beta=0.89\).
\section*{Problem 6}
\label{sec:org43c237b}

\textbf{Answer:} Yes\\

We begin by summarizing the information we were given. If we denote the value of
the assets by \(A\), then we have the following:
\begin{itemize}
\item equity is \(E=0.5 A\);
\item debt is \(D=0.5 A\);
\item cost of equity is \(R_e=0.1\);
\item cost of debt is \(R_d=0.05\);
\item tax rate is \(t=0\).
\end{itemize}
We can use the values above to compute the WACC. As explained in the lectures,
this quantity is given by
\begin{equation}
\mathrm{WACC}=\frac{E}{E+D}R_e+\frac{D}{E+D}(1-t)R_d.
\end{equation}
By substituting the known values into the RHS of this equation, we get
\begin{align}
  \begin{split}
    \mathrm{WACC}&=\frac{E}{E+D}R_e+\frac{D}{E+D}(1-t)R_d\\
    &=\frac{0.5 A}{0.5 A+0.5 A}R_e+\frac{0.5 A}{0.5 A+0.5 A}(1-0)R_d\\
    &=0.5 R_e+0.5 R_d\\
    &=0.5(R_e+R_d)\\
    &=0.5(0.1+0.05)\\
    &=0.5\cdot 0.15\\
    &=0.075.
  \end{split}
\end{align}
Then the cost of capital for this company is 7.5\%.\\
The next step is to use the WACC to compute the present value of \$600 million in
2 years. With the WACC as the discount rate, we get the following for this PV:
\begin{equation}
\mathrm{PV}=\frac{600}{(1+0.075)^2}=519.2.
\end{equation}
The present value of the income is \$519.2 million. Since this amount is greater
than the required investment, the company should take on the project.
\section*{Problem 7}
\label{sec:org349b788}

\textbf{Answer:} No\\

In this case, the present value corresponding to the income is
\begin{equation}
\mathrm{PV}=\frac{600}{(1+0.075)^3}=482.98.
\end{equation}
The PV of the income is \$482.98 million. Since this amount is less than the
required investment, the company should NOT take on the project.
\section*{Problem 8}
\label{sec:org00907eb}

\textbf{Answer:} 33.3\%\\

To solve this problem, we're going to use the accounting equation:
\begin{equation}
A=D+E,
\end{equation}
where \(A\) denotes the amount in assets, \(D\) denotes the debt amount, and
\(E\) denotes the equity amount. We know that \(D=0.25 A\). Hence:
\begin{align}
  \begin{split}
    A&=D+E\\
    A&=0.25 A+E\\
    A-0.25 A&=E\\
    E&=0.75 A
  \end{split}
\end{align}
The company's debt-equity ratio can now be computed as follows:
\begin{align}
  \begin{split}
    \frac{D}{E}&=\frac{0.25 A}{0.75 A}\\
    &=\frac{0.25}{0.75}\\
    &=\frac{\frac{1}{4}}{\frac{3}{4}}\\
    &=\frac{1}{3}
  \end{split}
\end{align}
As a percentage, this ratio is approximately 33.3\%.
\section*{Problem 9}
\label{sec:org0be6fbf}

\textbf{Answer:} 66.6\%\\

First, we use the debt-equity ratio to obtain an expression for \(D\):
\begin{equation}
\frac{D}{E}=0.5\qquad\Rightarrow\qquad D=0.5 E.
\end{equation}
Next, we substitute this expression into the accounting equation:
\begin{align}
  \begin{split}
    A&=D+E\\
    A&=0.5 E+E\\
    A&=1.5 E\\
    A&=\frac{3}{2}E\\
    E&=\frac{2}{3}A
  \end{split}
\end{align}
Therefore, two thirds of the company's assets are financed by equity. As a
percentage, this fraction is approximately 66.6\%.
\section*{Problem 10}
\label{sec:orgbb980ff}

\textbf{Answer:} 97\%\\

We begin by re-writing the definition of WACC. The goal is to make the
debt-equity ratio appear in that equation, and then solve for this ratio. This
can be done as follows:
\begin{align}
  \begin{split}
    \mathrm{WACC}&=\frac{E}{E+D}R_e+\frac{D}{E+D}(1-t)R_d\\
    (E+D)\mathrm{WACC}&=E R_e+D(1-t)R_d\\
    \frac{(E+D)\mathrm{WACC}}{E}&=\frac{E R_e+D(1-t)R_d}{E}\\
    \left(1+\frac{D}{E}\right)\mathrm{WACC}&=R_e+\frac{D}{E}(1-t)R_d\\
    \mathrm{WACC}+\frac{D}{E}\mathrm{WACC}&=R_e+\frac{D}{E}(1-t)R_d\\
    \frac{D}{E}\left[\mathrm{WACC}-(1-t)R_d\right]&=R_e-\mathrm{WACC}\\
    \frac{D}{E}&=\frac{R_e-\mathrm{WACC}}{\mathrm{WACC}-(1-t)R_d}\\
    \frac{D}{E}&=\frac{\mathrm{WACC}-R_e}{(1-t)R_d-\mathrm{WACC}}
  \end{split}
\end{align}
Now it's just a matter of using the values given in the problem statement:
\begin{itemize}
\item \(\mathrm{WACC}=0.098\);
\item \(R_e=0.13\);
\item \(R_d=0.065\).
\end{itemize}
The value of the tax rate wasn't specified. So we assume that \(t=0\). By
substituting the above values into our first equation for the debt-equity ratio,
we get
\begin{align}
  \begin{split}
    \frac{D}{E}&=\frac{R_e-\mathrm{WACC}}{\mathrm{WACC}-(1-t)R_d}\\
    &=\frac{R_e-\mathrm{WACC}}{\mathrm{WACC}-R_d}\\
    &=\frac{0.13-0.098}{0.098-0.065}\\
    &=0.97.
  \end{split}
\end{align}
Therefore, the debt-equity ratio for this company is 97\%.
\end{document}
